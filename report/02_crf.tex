\section{Conditional Random Fields}
\label{sec:crf}

\ac{CRF} are log-linear models for labelling and segmenting sequential data.
In general, log-linear models describe the probability of the label $y$ given the example $x$ as:
\begin{equation*}
    p(y|x;w) = \frac{exp \sum_{j = 1}^{J} F_{j}(x, y)}{Z(x, w)},
\end{equation*}
where $w$ is a vector of weights that describe the importance of each feature and $Z(x,w)$ is just a normalization factor.
$F_{j}$ is called a feature function, i.e. a measure of the compatibility between example $x$ and label $y$.
In \ac{CRF}, features are functions of the sentence $x$, the current label $y_{i}$, the previous label $y_{i-1}$ and the position $i$;
the output is usually binary.
An example of feature could be: ``1 if the word $x_i$ is capitalized and the label $y_i$ is \texttt{actor.name}''.

The power of \ac{CRF} is that anything can be a feature.
It is possible to use multiple features at the same time and combine them together to obtain new more complex features.
For instance, one could observe that some words are associated to a specific concept only if the phrase has a certain grammatical structure, and create some additional features that combines the \ac{POS} tags with the words in the sentence.
