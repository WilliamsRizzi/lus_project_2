\section{Conditional Random Fields}
\label{sec:crf}

\ac{CRF} are log-linear models for labelling and segmenting sequential data \cite{elkan}.
In general, log-linear models describe the probability of the label $y$ given the example $x$ as:
\begin{equation*}
    p(y|x;w) = \frac{exp \sum_{j = 1}^{J} F_{j}(x, y)}{Z(x, w)},
\end{equation*}
where $w$ is a vector of weights that describes the importance of each feature and $Z(x,w)$ is a normalization factor.
$F_{j}$ are feature functions, i.e. measures of the compatibility between example $x$ and label $y$.
In \ac{CRF}, the features are functions of the sentence $x$, the current label $y_{i}$, the previous label $y_{i-1}$ and the position $i$;
the output is usually binary, either $0$ or $1$.
An example of feature could be: ``$1$ if the word $x_i$ is capitalized and the label $y_i$ is \texttt{actor.name}, $0$ otherwise''.

In \ac{CRF} anything can be a feature.
It is possible to combine multiple features together to obtain new more complex ones or even generate new features from the data set.
For instance, if one observes that all words starting with a given prefix correspond to the concept $A$, one can add a feature that takes word prefixes into account.
