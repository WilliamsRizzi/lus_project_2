\documentclass[11pt,a4paper]{article}
\usepackage[hyperref]{acl2017}
\usepackage{times}
\usepackage{latexsym}
\usepackage{url}

% acronyms support
\usepackage{acronym}

% decent fonts
\usepackage[T1]{fontenc}
\usepackage{pifont}
\usepackage{microtype}

% nice looking tables and math
\usepackage{booktabs}
\usepackage{amsmath}

% images
\usepackage{graphicx}

% cref
\usepackage[noabbrev,capitalise]{cleveref}

% list of acronyms
\acrodef{SLU}{Spoken Language Understanding}
\acrodef{WFST}{Weighted Finite State Transducers}
\acrodef{CRF}{Conditional Random Fields}
\acrodef{POS}{Part of Speech}
\acrodef{IOB}{Inside, Outside, Beginning}

% include name and email
\aclfinalcopy

\title{
  Concept Sequence Tagging for a Movie Domain with CRF \\
  Language Understanding Systems, Second Project
} 

\author{Davide Pedranz \\
  Mat. number 189295 \\
  {\tt davide.pedranz@studenti.unitn.it}
}

\date{28 May 2017}

\begin{document}
\maketitle

\begin{abstract}
Concept Sequence Tagging is a fundamental task for any \ac{SLU} system.
Concepts can be used to understand the semantic of user's requests and give appropriate replies.
In this project, we trained \ac{CRF} classifiers to extract concepts from sentences taken from the movie domain.
Then, we compared the performances with the \ac{WFST} generative models developed in the mid-term project.
\end{abstract}

% content
\section{Introduction}
\label{sec:introduction}

Concept tagging can be defined as the extraction of concepts out of a given word sequence,
where a concept represents the smallest unit of meaning that is relevant for a specific task.
The extracted concepts can be used by the following blocks of a \ac{SLU} pipeline, like a dialog management system, to understand the semantic of user's requests and build a appropriate replies.

In this work, we trained \ac{CRF} classifiers to extract concepts sentences in the movie domain, build using
the \texttt{CRF++}\footnote{\url{https://taku910.github.io/crfpp/}} toolkit.
We will give a brief description of \ac{CRF} models.
Then, we will present the data set and the features used for the classification.
We will present the different experiments performed and the obtained results.
Finally, we will compare the \ac{CRF} to the \ac{WFST} models of the mid-term project.

\section{Conditional Random Fields}
\label{sec:crf}

\ac{CRF} are log-linear models for labelling and segmenting sequential data \cite{elkan}.
In general, log-linear models describe the probability of the label $y$ given the example $x$ as:
\begin{equation*}
    p(y|x;w) = \frac{exp \sum_{j = 1}^{J} F_{j}(x, y)}{Z(x, w)},
\end{equation*}
where $w$ is a vector of weights that describes the importance of each feature and $Z(x,w)$ is a normalization factor.
$F_{j}$ are feature functions, i.e. measures of the compatibility between example $x$ and label $y$.
In \ac{CRF}, the features are functions of the sentence $x$, the current label $y_{i}$, the previous label $y_{i-1}$ and the position $i$;
the output is usually binary, either $0$ or $1$.
An example of feature could be: ``$1$ if the word $x_i$ is capitalized and the label $y_i$ is \texttt{actor.name}, $0$ otherwise''.

In \ac{CRF} anything can be a feature.
It is possible to combine multiple features together to obtain new more complex ones or even generate new features from the data set.
For instance, if one observes that all words starting with a given prefix correspond to the concept $A$, one can add a feature that takes word prefixes into account.

\section{Data Set}
\label{sec:dataset}

The data set used is a corpora of sentences from the movie domain.
The provided features are: words, \ac{POS} tags and words stems.
The aim is to extract concepts such as the actor name or the movie release date in the \ac{IOB} format.
Please checkout the mid-term project for a detailed description of the data set and the concepts distribution.

\subsection{Preprocessing}
Before training the classifiers, some additional features were computed from the available data:
\begin{itemize}
    \item the word's stem is capitalized;
    \item the word is a language;
    \item the word prefix (of 1, 2 and 3 letters);
    \item the word suffix (of 1, 2 and 3 letters).
\end{itemize}

\section{Experiments}
\label{sec:experiments}

We performed the following trials:
\begin{enumerate}
    \item Train a simple model using only the words
    \item Add additional features among the original ones
    \item Create new features as a combination of the others
    \item Use a genetic algorithm to automatically create and select the best features
\end{enumerate}

\subsection{Model 1}
\label{subsection:1}
The first trial was to try to classify the concepts using only the words of the sentence.
We experimented using different window sizes, i.e. creating a feature function for the current word and the preceding and following $n$ words, with $n \in [0,8]$.
In addition, we tried to enable the bigram templates of \texttt{CRF++}, i.e. considering also the label of the previous word to classify the current one. 

\begin{table}[t!]
	\centering
    \begin{tabular}{ c c c }
    	\toprule
    		\multicolumn{1}{c}{n} & \multicolumn{1}{c}{F1, no bigram} & \multicolumn{1}{c}{F1, bigram} \\
    	\midrule
            0 & 51.60\% & 72.11\% \\
1 & 69.18\% & 79.71\% \\
2 & 75.52\% & 81.39\% \\
3 & 77.21\% & 80.99\% \\
4 & 77.69\% & 80.21\% \\
5 & 76.76\% & 80.23\% \\
6 & 76.70\% & 80.04\% \\

    	\bottomrule
	\end{tabular}
    \caption{Performances of the models using only the words as a feature. For each window size, the model is trained with and without the bigram template. In all cases, the bigram template helps to get higher performances.}
	\label{tab:words}
\end{table}

\cref{tab:words} summaries the performances of the different models.
In all cases, the bigram template helps to get significantly higher performances.
The best model is the one with window size $2$, using bigrams.
The performances degrades for higher window sizes.
Probably the words which are very far away from the current one have very little effect on the semantic of the current one.
On the other side, the concept of the previous word seems to be very important to better classify the current one.

\section{Comparison}
\label{sec:comparison}

\subsection{Baseline}
The best \ac{WFST} model in the mid-term project used a 4-gram language model for the concept tags and achieved a F1 score of $82.74\%$.
The baseline \ac{CRF} model described in \cref{subsection:words} achieved a F1 score of $81.39\%$.
Both models used only the words as a feature.
\ac{CRF} models, due to the \texttt{CRF++} limitations and the model complexity, can only take into account the current and the previous label for the classification.
In the \ac{WFST} model, this would correspond to a 2-gram language model for the concepts, which performed $79.45\%$ F1 score.
\ac{WFST} models are simpler to build and can thus use more complex language models that can outperform more complex generative methods.

\subsection{Complex Models}
The most advanced \ac{CRF} models trained described in \cref{subsection:additional} achieved a F1 score of $83.57\%$.
The model combines words, POS tags, stems, prefixes and radixes.
This shows how discriminative models can exploit additional information that can not be easily modeled with generative models.
Thanks to this additional information, \ac{CRF} outperformed \ac{WFST} on the concept tagging task.

\begin{table}[h]
	\centering
    \begin{tabular}{ c c c c }
    	\toprule
    		\multicolumn{1}{c}{algorithm} & \multicolumn{1}{c}{prec.} & \multicolumn{1}{c}{rec.} & \multicolumn{1}{c}{F1} \\
    	\midrule
            WFST (4-gram) & 82.44\% & 83.04\% & 82.74\% \\
            CRF (manual) & 86.72\% & 80.66\% & 83.57\% \\
			CRF (genetic) & 87.52\% & 81.03\% & 84.15\% \\
    	\bottomrule
	\end{tabular}
    \caption{Comparison of the performances of the best models for \ac{WFST} and \ac{CRF}.}
	\label{tab:best}
\end{table}

\cref{tab:best} compares the performances of best \ac{WFST} and \ac{CRF} models.
We can notice that the \ac{WFST} model has a significantly higher recall than both \ac{CRF} models.
\ac{WFST} tends to be more conservative in the classification, probably due to the more complex language model (4-grams) than tends to prevent false positives.
On the other hand, \ac{CRF} models exploit multiple features and are able to find more concepts, at the cost of some extra false positive.

\subsection{Training Complexity}
\ac{WFST} are very easy to train, since the training only requires to estimate some probability distributions from the training data.
This can be done by simply compute frequencies and normalizing them to probabilities.
Some effort should be put into the choice of the meta-parameters such as the window size for the n-gram language model and the smoothing method.

In contrast, \ac{CRF} require to manually specify the features to use.
As described in \cref{sec:experiments}, this can be very difficult due to the exponential number of possible combinations.
The problem can be partially solved using optimization techniques such as genetic algorithms.
These methods automatically pick the best features combination, at the cost of some additional computations in the training phase.
In our case, the genetic algorithm managed to improved significantly the model performances, as described in \cref{subsection:genetic}.

\section{Conclusion}
\label{sec:conclusion}
The most advances \ac{CRF} models outperformed the \ac{WFST} models developed in the mid-term project, thanks to the discriminative nature that allowed to easily combine multiple features together.
On the other side, \ac{CRF} resulted much more difficult to train than \ac{WFST} due to the exponential number of possible feature functions combinations.
Some heuristics can be used to come up with good features, then optimization methods can help to select the best ones.
In particular, our genetic algorithm managed to generate a population of \ac{CRF} models which outperformed the best model trained by hand, reaching the F1 score of $84.15\%$.


% bibliography
\bibliographystyle{acl_natbib}
\bibliography{references}

\end{document}
