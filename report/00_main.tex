\documentclass[11pt,a4paper]{article}
\usepackage[hyperref]{acl2017}
\usepackage{times}
\usepackage{latexsym}
\usepackage{url}

% acronyms support
\usepackage{acronym}

% decent fonts
\usepackage[T1]{fontenc}
\usepackage{pifont}
\usepackage{microtype}

% nice looking tables and math
\usepackage{booktabs}
\usepackage{amsmath}

% images
\usepackage{graphicx}

% cref
\usepackage[noabbrev,capitalise]{cleveref}

% list of acronyms
\acrodef{SLU}{Spoken Language Understanding}
\acrodef{WFST}{Weighted Finite State Transducers}
\acrodef{CRF}{Conditional Random Fields}
\acrodef{POS}{Part of Speech}
\acrodef{IOB}{Inside, Outside, Beginning}
\acrodef{OOV}{Out of Vocabulary}
\acrodef{FSA}{Finite State Acceptor}
\acrodef{FST}{Finite State Transducer}

% math commands
\DeclareMathOperator*{\argmax}{arg\!\max}

% include name and email
\aclfinalcopy

\title{
  Concept Sequence Tagging for a Movie Domain with CRF \\
  Language Understanding Systems, Second Project
} 

\author{Davide Pedranz \\
  Mat. number 189295 \\
  {\tt davide.pedranz@studenti.unitn.it}
}

\date{28 May 2017}

\begin{document}
\maketitle

\begin{abstract}
Concept Sequence Tagging is a fundamental for any \ac{SLU} system.
The concepts extracted from the user sentences can be used to understand the user's request and prepare the appropriate reply.
In this project, we trained discriminative \ac{CRF} classifiers to extract concepts from sentences taken from the movie domain and compared the performances with the \ac{WFST} generative models developed in the mid-term project.
\end{abstract}

\section{Introduction}
\label{sec:introduction}

Concept tagging can be defined as the extraction of concepts out of a given word sequence,
where a concept represents the smallest unit of meaning that is relevant for a specific task.
The extracted concepts can be used by the following blocks of a \ac{SLU} pipeline, like a dialog management system, to understand the semantic of user's requests and build a appropriate replies.

In this report, we will train \ac{CRF} classifiers to extract concepts sentences in the movie domain, build using
the \texttt{CRF++}\footnote{\url{https://taku910.github.io/crfpp/}} toolkit.
We will give a brief description \ac{CRF} models.
Then, we will present the data set and the available features.
We will present the different experiments performed and the obtained results.
Finally, we will compare the \ac{CRF} to the \ac{WFST} models of the mid-term project.

\section{Conditional Random Fields}
\label{sec:crf}

\ac{CRF} are log-linear models for labelling and segmenting sequential data \cite{elkan}.
In general, log-linear models describe the probability of the label $y$ given the example $x$ as:
\begin{equation*}
    p(y|x;w) = \frac{exp \sum_{j = 1}^{J} F_{j}(x, y)}{Z(x, w)},
\end{equation*}
where $w$ is a vector of weights that describe the importance of each feature and $Z(x,w)$ is just a normalization factor.
$F_{j}$ is called a feature function, i.e. a measure of the compatibility between example $x$ and label $y$.
In \ac{CRF}, features are functions of the sentence $x$, the current label $y_{i}$, the previous label $y_{i-1}$ and the position $i$;
the output is usually binary.
An example of feature could be: ``$1$ if the word $x_i$ is capitalized and the label $y_i$ is \texttt{actor.name}''.

The power of \ac{CRF} is that anything can be a feature.
It is possible to use multiple features at the same time and combine them together to obtain new more complex features.
For instance, one could observe that some words are associated to a specific concept only if the phrase has a certain grammatical structure, and create some additional features that combines the \ac{POS} tags with the words in the sentence.

\section{Data Set}
\label{sec:dataset}

The data set used is a corpora of sentences from the movie domain.
The provided features are: words, \ac{POS} tags and words stems.
The aim is to extract concepts such as the actor name or the movie release date in the \ac{IOB} format.
Please checkout the mid-term project for a detailed description of the data set and the concepts distribution.

\subsection{Preprocessing}
Before training the classifiers, some additional features were computed from the available data:
\begin{itemize}
    \item the word's stem is capitalized;
    \item the word is a language;
    \item the word prefix (of 1, 2 and 3 letters);
    \item the word suffix (of 1, 2 and 3 letters).
\end{itemize}

% \input{04_implementation}
% \input{05_improvements}
% \input{06_performances}
% \input{07_conclusions}

\end{document}
