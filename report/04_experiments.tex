\section{Experiments}
\label{sec:experiments}

We performed the following trials:
\begin{enumerate}
    \item Train a simple model using only the words
    \item Add additional features among the original ones
    \item Create new features as a combination of the others
    \item Use a genetic algorithm to automatically create and select the best features
\end{enumerate}

\subsection{Model 1}
\label{subsection:1}
The first trial was to try to classify the concepts using only the words of the sentence.
We experimented using different window sizes, i.e. creating a feature function for the current word and the preceding and following $n$ words, with $n \in [0,8]$.
In addition, we tried to enable the bigram templates of \texttt{CRF++}, i.e. considering also the label of the previous word to classify the current one. 

\begin{table}[t!]
	\centering
    \begin{tabular}{ c c c }
    	\toprule
    		\multicolumn{1}{c}{n} & \multicolumn{1}{c}{F1, no bigram} & \multicolumn{1}{c}{F1, bigram} \\
    	\midrule
            0 & 51.60\% & 72.11\% \\
1 & 69.18\% & 79.71\% \\
2 & 75.52\% & 81.39\% \\
3 & 77.21\% & 80.99\% \\
4 & 77.69\% & 80.21\% \\
5 & 76.76\% & 80.23\% \\
6 & 76.70\% & 80.04\% \\

    	\bottomrule
	\end{tabular}
    \caption{Performances of the models using only the words as a feature. For each window size, the model is trained with and without the bigram template. In all cases, the bigram template helps to get higher performances.}
	\label{tab:words}
\end{table}

\cref{tab:words} summaries the performances of the different models.
In all cases, the bigram template helps to get significantly higher performances.
The best model is the one with window size $2$, using bigrams.
The performances degrades for higher window sizes.
Probably the words which are very far away from the current one have very little effect on the semantic of the current one.
On the other side, the concept of the previous word seems to be very important to better classify the current one.
